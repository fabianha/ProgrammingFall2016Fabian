\nonstopmode{}
\documentclass[a4paper]{book}
\usepackage[times,inconsolata,hyper]{Rd}
\usepackage{makeidx}
\usepackage[utf8]{inputenc} % @SET ENCODING@
% \usepackage{graphicx} % @USE GRAPHICX@
\makeindex{}
\begin{document}
\chapter*{}
\begin{center}
{\textbf{\huge Package `FabiansPackage'}}
\par\bigskip{\large \today}
\end{center}
\begin{description}
\raggedright{}
\inputencoding{utf8}
\item[Title]\AsIs{What the Package Does (one line, title case)}
\item[Version]\AsIs{0.0.0.9000}
\item[Description]\AsIs{This package finds the highest number in a vector. Define the vector as a variable before inserting it as as argument).}
\item[Depends]\AsIs{R (>= 3.3.1)}
\item[License]\AsIs{MIT + file LICENSE?}
\item[Encoding]\AsIs{UTF-8}
\item[LazyData]\AsIs{true}
\item[RoxygenNote]\AsIs{5.0.1}
\item[NeedsCompilation]\AsIs{no}
\item[Author]\AsIs{fabian hafner [aut, cre]}
\item[Maintainer]\AsIs{fabian hafner }\email{fabian.hafner@sciencespo.fr}\AsIs{}
\end{description}
\Rdcontents{\R{} topics documented:}
\inputencoding{utf8}
\HeaderA{vector\_max}{Finding the highest number in a vector}{vector.Rul.max}
%
\begin{Description}\relax
This function finds the biggest number in a vector
\end{Description}
%
\begin{Usage}
\begin{verbatim}
vector_max(vector)
\end{verbatim}
\end{Usage}
%
\begin{Examples}
\begin{ExampleCode}
vector_max()
\end{ExampleCode}
\end{Examples}
\printindex{}
\end{document}
